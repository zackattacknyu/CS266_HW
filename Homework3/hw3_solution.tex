\documentclass[11pt,psfig]{article}
\usepackage{epsfig}
\usepackage{times}
\usepackage{amssymb}
\usepackage{float}

\newcount\refno\refno=1
\def\ref{\the\refno \global\advance\refno by 1}
\def\ux{\underline{x}}
\def\uw{\underline{w}}
\def\bw{\underline{w}}
\def\ut{\underline{\theta}}
\def\umu{\underline{\mu}} 
\def\bmu{\underline{\mu}} 
\def\be{p_e^*}
\newcount\eqnumber\eqnumber=1
\def\eq{\the \eqnumber \global\advance\eqnumber by 1}
\def\eqs{\eq}
\def\eqn{\eqno(\eq)}

 \pagestyle{empty}
\def\baselinestretch{1.1}
\topmargin1in \headsep0.3in
\topmargin0in \oddsidemargin0in \textwidth6.5in \textheight8.5in
\begin{document}
\setlength{\parskip}{1.2ex plus0.3ex minus 0.3ex}


\thispagestyle{empty} \pagestyle{myheadings} \markright{G}



\title{CS 266 Homework 3}
\author{Zachary DeStefano, PhD Student, 15247592}
\date{Due Date: April 24}

\maketitle

\vfill\eject

\section*{Problem 3.11}

Give an efficient algorithm to determine whether a polygon P with n
vertices is monotone with respect to some line, not necessarily a horizontal
or vertical one.\\
\\
For this algorithm, we will use a version of the plane sweep algorithm. We will sweep a horizontal line at each vertex and if the intersection is more than just two points, a line segment, or empty, then it is not monotone. \\
\\
Here is the algorithm:\\
1. Sort the vertices by y-coordinate\\
2. Make an interval tree data structure for the $y_{min}$ and $y_{max}$ coordinates of each of the segments. \\
\begin{verbatim}
http://en.wikipedia.org/wiki/Interval_tree#Centered_interval_tree
\end{verbatim}
3. For each vertex v, do the following:\\
- Sweep a horizontal line at its y-coordinate\\
- If the number of other segments or points with that same y-coordinate is more than 1\\
or there is at least one other intersection but v is part of a horizontal segment, then\\
declare that P is not monotone and exit.\\
(TODO: Detail the data structure to be used here)\\
4. If P has not been declared non-monotone, then P is monotone\\
\\
Correctness:\\
All the parts where the it will not be polygon will be at event points, which are the vertices. (TODO: Prove this)\\
\\
Running time: \\
Step 1 will take O(n log n) time. \\
Constructing an interval tree for Step 2 is O(n log n) time. \\
There are n vertices to test in the worst case. \\
For each vertex, the query will take O($log n + 2$) time since we are requesting 2 results. \\
Thus the total running time ends up being O(n log n)\\

\newpage

\section*{Problem 3.14}

Given a simple polygon P with n vertices and a point p inside it, show
how to compute the region inside P that is visible from p.\\
\\
In the following figure, the visible region is the triangles with an X inside them. \\
\begin{figure}[H]
\centering
\includegraphics[height=4in]{visible_regions.jpg}
\caption{Parts of polygon visible from point p}
\end{figure}

The following procedure will be used:\\
For each vertex v with unobstructed view of p:\\
- Construct line segment from p, then passing through v, and ending at the next line segment in that direction. \\
- All the triangles that have just been constructed that are around p are the visible region. 

\newpage

\section*{Problem 15.2}

Algorithm VISIBILITYGRAPH calls algorithm VISIBLEVERTICES with
each obstacle vertex. VISIBLEVERTICES sorts all vertices around its
input point. This means that n cyclic sortings are done, one around each
obstacle vertex. In this chapter we simply did every sort in O(nlogn)
time, leading to O($n^2$ logn)time for all sortings. Show that this can be
improved to O($n^2$) time using dualization (see Chapter 8). Does this
improve the running time of VISIBILITYGRAPH?\\
\\
Idea:\\
Take the vertices and dualize them. \\
Construct the arrangement of the resulting lines. \\
Something in the arrangment tells you about how to construct the visibility graph

\newpage

\section*{Problem 15.4}

What is the maximal number of shortest paths connecting two fixed
points among a set of n triangles in the plane?\\
\\
The visibility graph will have $V = (3n+2)$.\\
For each vertex $v$ we have $|v| \leq 3n + 2$ for the number of vertices it is connected to. \\
The max number of paths is the max number of sequences of such vertices. \\
We need to start with $s$ and end with $t$ but the vertices in the middle do not matter. \\
We end up with $(3n)!$ paths possible. 


\end{document}








