\documentclass[11pt,psfig]{article}
\usepackage{epsfig}
\usepackage{times}
\usepackage{amssymb}
\usepackage{float}

\newcount\refno\refno=1
\def\ref{\the\refno \global\advance\refno by 1}
\def\ux{\underline{x}}
\def\uw{\underline{w}}
\def\bw{\underline{w}}
\def\ut{\underline{\theta}}
\def\umu{\underline{\mu}} 
\def\bmu{\underline{\mu}} 
\def\be{p_e^*}
\newcount\eqnumber\eqnumber=1
\def\eq{\the \eqnumber \global\advance\eqnumber by 1}
\def\eqs{\eq}
\def\eqn{\eqno(\eq)}

 \pagestyle{empty}
\def\baselinestretch{1.1}
\topmargin1in \headsep0.3in
\topmargin0in \oddsidemargin0in \textwidth6.5in \textheight8.5in
\begin{document}
\setlength{\parskip}{1.2ex plus0.3ex minus 0.3ex}


\thispagestyle{empty} \pagestyle{myheadings} \markright{G}



\title{CS 266 Homework 6}
\author{Zachary DeStefano, PhD Student, 15247592}
\date{Due Date: May 22}

\maketitle

\vfill\eject

\section*{Problem 6.13}

As a vertical line sweeps across, it will be making a trapezoid. \\
At a left endpoint, there are three trapezoids:\\
1. One already existing to the right\\
2. One being made above existing segment\\
3. One being made below existing segment\\
\\
There are n segments and thus 2n endpoints. Considering that there are 3 trapezoids at each endpoint, this makes 6n trapezoids. However, we are double counting each trapezoid, making it 3n trapezoids. We are not however double counting in the case of the first endpoint that is considered because the trapezoid before it was not created by another endpoint, thus that adds 1 trapezoid. This means that there are at most $3n+1$ trapezoids for n line segments.

\newpage

\section*{Problem 6.15}

 Although we have started with the point location problem on the surface
of the earth, we have only treated planar point location. But the earth is
a globe. How would you define a spherical subdivision—a subdivision
of the surface of a sphere? Give a point location structure for such a
subdivision.\\
\\
Divide the sphere into cross-sections by x-coordinate. The top and bottom ones would be degenerate ones. You then divide up the cross-sections. \\
\\
Identically, each point on the surface of the sphere can be described by two angles $(\theta,\phi)$. You can take these coordinates for each point and put them into a 2-D space and then do the point location map. Vertical segments will correspond to the cross sections described above. 
%\begin{figure}[H]
%\centering
%\includegraphics[height=4in]{prob1plot.jpg}
%\caption{Probability of Class Labels with decision boundaries marked}
%\end{figure}


\end{document}








