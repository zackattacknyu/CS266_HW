\documentclass[11pt,psfig]{article}
\usepackage{epsfig}
\usepackage{times}
\usepackage{amssymb}
\usepackage{float}

\newcount\refno\refno=1
\def\ref{\the\refno \global\advance\refno by 1}
\def\ux{\underline{x}}
\def\uw{\underline{w}}
\def\bw{\underline{w}}
\def\ut{\underline{\theta}}
\def\umu{\underline{\mu}} 
\def\bmu{\underline{\mu}} 
\def\be{p_e^*}
\newcount\eqnumber\eqnumber=1
\def\eq{\the \eqnumber \global\advance\eqnumber by 1}
\def\eqs{\eq}
\def\eqn{\eqno(\eq)}

 \pagestyle{empty}
\def\baselinestretch{1.1}
\topmargin1in \headsep0.3in
\topmargin0in \oddsidemargin0in \textwidth6.5in \textheight8.5in
\begin{document}
\setlength{\parskip}{1.2ex plus0.3ex minus 0.3ex}


\thispagestyle{empty} \pagestyle{myheadings} \markright{Homework
1: CS 266, Computational Geometry: Spring 2014}



\title{CS 266 Homework 1}
\author{Zachary DeStefano, 15247592}
\date{Due Date: April 10, 2014}

\maketitle

\vfill\eject

\section*{Problem 1.6}

\subsection*{Problem 1.6 Part A}

Since the convex hull is a convex set containing all the points, it holds that the convex hull of the 2n endpoints is a convex set that contains all the line segments. Due to the convexity, all the line segments will be in the convex hull. 
\\
\\
Since the endpoints are on the line, a convex set of the n line segments must contain all the 2n endpoints. Since the convex hull of 2n endpoints is the smallest convex set that contains all the endpoints, it is thus the smallest convex set that contains all the n line segments, thus it is the convex hull of the line segments. 

\subsection*{Problem 1.6 Part B}

Let P be a non-convex polygon. Algorithm for finding the convex hull of P in O(n) time:
\\
1. Find left-most point and right-most point in polygon. 
2. To compute the upper hull, build up the line segments using the same algorithm as ConvexHull and use the vertices in lexicographic order of the polygon
3. Add or delete vertices depending on whether the slope is decreasing. 
TODO: Improve this description

\section*{Problem 1.10}

Let S be a set of n possibly intersecting unit circles

\subsection*{Problem 1.10a}

Proof that the convex hull of S consists of straight lines and pieces of circles in S. 
\\
A circle can be considered an infinite sided polygon. With this in mind, consider a set of regular k-gons with radius of 1. The convex hull will be convex set that contains all the vertices of the k-gons. There will thus be straight lines between each of the k-gons and parts of the hull be along the k-gon. 
\\
A circle is just a k-gon where $k=\infty$ thus the above holds for a circle. 

\subsection*{Problem 1.10b}

Assume that the convex hull appears on the boundary of a circle twice. There are two cases to consider.\\
Case 1: The convex hull is traveling along the circle and then goes inside the circle before going back to the boundary. \\
Case 2: The convex hull is traveling along the circle and then goes outside the circle before going back to the boundary.\\
\\
In case 1, you can connect the two end points and you will connect two points in the hull but part of the line will be outside the hull, thus it won't be a true convex hull, a contradiction\\
\\
In case 2, for the upper hull, the slope of the hull will have to increase to leave the hull. For the lower hull, the slope will have to decrease. This violates a basic constraint of convex hull, so we have a contradiction. \\
\\
Since we have a contradiction for both cases where a convex hull appears in two different places on the boundary of a circle, it is impossible for that to occur, thus the convex hull is either on the boundary once or not at all. 

\subsection*{Problem 1.10c}

Take the convex hull of S'. 
\\
Assume that a point is in the convex hull of S' but the corresponding circle is not on the boundary of the convex hull. This turns out to be a contradiction as there would be some other unaccounted for circle. \\
Assume that a circle is in the convex hull of S but its center is not in S', then there is some circle that was not considered and this eventually leads to a contradiction
\\
TODO: Improve the proof

\subsection*{Problem 1.10d}

The idea of the algorithm is that we will find the convex hull of the centers and then use that to get the total convex hull. Here is the algorithm:\\
1. Find the convex hull of the centers\\
2. For every line pq in that convex hull\\
			Find line perpendicular to pq that passes through p and line perpendicular to pq that passes through q, call them p1 and q1\\
			Find the intersections of the p circle with p1 and intersections of q circle with q1, call them p1*, p2*, q1*, q2*. Make sure p1*, q1* are on the same side of the pq line segment. \\
			If there are only two circles, then connect p1* and q1* in one segment and then p2* and q2* in another segment. The hull is then p1*,q1*,q2*,p2*,p1*\\
			If there are more than two circles, then make p1*,q1* the pair that outside the convex hull of the centers. Connect p1* and q1* and make that part of the hull\\
3. You now have circles and straight lines. Travel along the straight lines and add it to hull. When you reach a circle, you have to decide which part of the boundary to add to the hull\\
			Add the part of that circle that will make slope decrease if doing the upper hull.\\
			Add the part to make the slope increase if doing the lower hull.

\subsection*{Problem 1.10e}

TODO: Figure out algorithm for hull if circles have different radii


\section*{Problem 8.2}

\subsection*{Problem 8.2a}

The dual of a collection of points inside a triangle with vertices p,q,r will be a collection of lines from the union of the double wedges for pq, qr, and pr. 
\\
When you take $p*, q*, r*$, the division of the plane forms 6 wedges. There are 7 faces when you include the inner triangle formed. In the following figure the dual ends up being everything except section 5 and 6. It will end up looking like a double wedge with an extra triangle in the middle. 

\begin{figure}[H]
\centering
\includegraphics[height=4in]{cs266dual.jpg}
\caption{Illustration of the wedges}
\end{figure}

\subsection*{Problem 8.2b}

In the illustration, the left-right double wedge is formed by the line segment. The dual lines all meet in that point since the primal points are co-linear. If we take the entire line formed by pq, the dual of that is the union of all lines that pass through that point. For the top-bottom double wedge, we would then want that whole union minus the left-right double wedge. \\
\\
For a line segment $pq$ it can be described as ${tp + (1-t)q}$ for $0 \leq t \leq 1$. The two rays we want are the whole line minus that segment, thus it is ${tp + (1-t)q}$ for $t < 0$ and $t > 1$. 

\section*{Problem 8.6}

The duality transform is incidence-preserving thus the dual of the problem is whether any of the m points in the dual of L lie on any of the n lines in the dual of S. 


\end{document}








